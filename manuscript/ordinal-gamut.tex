\documentclass[]{article}
\usepackage{lmodern}
\usepackage{amssymb,amsmath}
\usepackage{ifxetex,ifluatex}
\usepackage{fixltx2e} % provides \textsubscript
\ifnum 0\ifxetex 1\fi\ifluatex 1\fi=0 % if pdftex
  \usepackage[T1]{fontenc}
  \usepackage[utf8]{inputenc}
\else % if luatex or xelatex
  \ifxetex
    \usepackage{mathspec}
  \else
    \usepackage{fontspec}
  \fi
  \defaultfontfeatures{Ligatures=TeX,Scale=MatchLowercase}
\fi
% use upquote if available, for straight quotes in verbatim environments
\IfFileExists{upquote.sty}{\usepackage{upquote}}{}
% use microtype if available
\IfFileExists{microtype.sty}{%
\usepackage{microtype}
\UseMicrotypeSet[protrusion]{basicmath} % disable protrusion for tt fonts
}{}
\usepackage[margin=1in]{geometry}
\usepackage{hyperref}
\hypersetup{unicode=true,
            pdftitle={Leveraging Family History in Case-Control Analyses of Rare Variation},
            pdfauthor={Claudia R. Solis-Lemus\^{}\{1*\}, S. Taylor Fischer\^{}\{2*\}, Andrei Todor\^{}1, Cuining Liu\^{}3, Elizabeth J. Leslie\^{}1, David J. Cutler\^{}1, Debashis Ghosh\^{}3, Michael P. Epstein\^{}1},
            pdfborder={0 0 0},
            breaklinks=true}
\urlstyle{same}  % don't use monospace font for urls
\usepackage{natbib}
\bibliographystyle{apalike}
\usepackage{longtable,booktabs}
\usepackage{graphicx,grffile}
\makeatletter
\def\maxwidth{\ifdim\Gin@nat@width>\linewidth\linewidth\else\Gin@nat@width\fi}
\def\maxheight{\ifdim\Gin@nat@height>\textheight\textheight\else\Gin@nat@height\fi}
\makeatother
% Scale images if necessary, so that they will not overflow the page
% margins by default, and it is still possible to overwrite the defaults
% using explicit options in \includegraphics[width, height, ...]{}
\setkeys{Gin}{width=\maxwidth,height=\maxheight,keepaspectratio}
\IfFileExists{parskip.sty}{%
\usepackage{parskip}
}{% else
\setlength{\parindent}{0pt}
\setlength{\parskip}{6pt plus 2pt minus 1pt}
}
\setlength{\emergencystretch}{3em}  % prevent overfull lines
\providecommand{\tightlist}{%
  \setlength{\itemsep}{0pt}\setlength{\parskip}{0pt}}
\setcounter{secnumdepth}{5}
% Redefines (sub)paragraphs to behave more like sections
\ifx\paragraph\undefined\else
\let\oldparagraph\paragraph
\renewcommand{\paragraph}[1]{\oldparagraph{#1}\mbox{}}
\fi
\ifx\subparagraph\undefined\else
\let\oldsubparagraph\subparagraph
\renewcommand{\subparagraph}[1]{\oldsubparagraph{#1}\mbox{}}
\fi

%%% Use protect on footnotes to avoid problems with footnotes in titles
\let\rmarkdownfootnote\footnote%
\def\footnote{\protect\rmarkdownfootnote}

%%% Change title format to be more compact
\usepackage{titling}

% Create subtitle command for use in maketitle
\newcommand{\subtitle}[1]{
  \posttitle{
    \begin{center}\large#1\end{center}
    }
}

\setlength{\droptitle}{-2em}

  \title{Leveraging Family History in Case-Control Analyses of Rare Variation}
    \pretitle{\vspace{\droptitle}\centering\huge}
  \posttitle{\par}
    \author{Claudia R. Solis-Lemus\(^{1*}\), S. Taylor Fischer\(^{2*}\), Andrei
Todor\(^1\), Cuining Liu\(^3\), Elizabeth J. Leslie\(^1\), David J.
Cutler\(^1\), Debashis Ghosh\(^3\), Michael P. Epstein\(^1\)}
    \preauthor{\centering\large\emph}
  \postauthor{\par}
    \date{}
    \predate{}\postdate{}
  
\DeclareMathOperator{\logit}{logit}
\DeclareMathOperator{\trace}{trace}
\usepackage[dvipsnames]{xcolor}
\newcommand{\missing}[1]{\textcolor{red}{\textbf{#1}}}
\newcommand{\revision}[1]{\textcolor{blue}{#1}}
\usepackage{setspace}

\doublespacing
\usepackage{array,ragged2e}
\usepackage{float}
\def\HS{\rule[2cm]{0pt}{0cm}}
\usepackage{lineno}
\linenumbers

\begin{document}
\maketitle

\begin{tabular}{l}
$*$ Joint first author \\[-0.25cm]
$^1$ Department of Human Genetics, Emory University, Atlanta, GA \\[-0.25cm]
$^2$ Department of Biostatistics and Bioinformatics, Emory University, Atlanta, GA \\[-0.25cm]
$^3$ Department of Biostatistics and Informatics, University of Colorado, Aurora, CO
\end{tabular}

\pagebreak

\textbf{Short title:} Family History in Case-Control Studies

\textbf{Key words:} rare variant, gene mapping, complex human traits

\begin{tabular}{l}
Address for correspondence:\\[-0.25cm]
Dr. Michael Epstein\\[-0.25cm]
Department of Human Genetics\\[-0.25cm]
Emory University School of Medicine,\\[-0.25cm]
Atlanta, GA, 30030\\[-0.25cm]
Email: mpepste@emory.edu\\[-0.25cm]
Phone: (404) 712-8289
\end{tabular}

\pagebreak

\begin{center}
\textbf{Abstract} 
\end{center}

Standard methods for case-control association studies of rare variation
often treat disease outcome as a dichotomous phenotype. However, both
theoretical and experimental studies have demonstrated that subjects
with a family history of disease can be enriched for risk variation
relative to subjects without such history. Assuming family history
information is available, this observation motivates the idea of
replacing the standard dichotomous outcome variable used in case-control
studies with a more informative ordinal outcome variable that
distinguishes controls (\(0\)), sporadic cases (\(1\)), and cases with a
family history (\(2\)), with the expectation that we should observe
increasing number of risk variants with increasing category of the
ordinal variable. To leverage this expectation, we propose a novel
rare-variant association test that incorporates family history
information based on our previous GAMuT framework \citep{Broadaway2016}
for rare-variant association testing of multivariate phenotypes. We use
simulated data to show that, when family history information is
available, our new method outperforms standard rare-variant association
methods like burden and SKAT tests that ignore family history. We
further illustrate our method using a rare-variant study of cleft lip
and palate.

\pagebreak

\hypertarget{introduction}{%
\section{Introduction}\label{introduction}}

Sequencing and exome-chip technologies facilitate the discovery of rare
genetic variation influencing complex diseases. Many rare-variant
association studies of complex diseases now exist with most studies
employing traditional case-control sampling designs for analysis
\citep{DeRubeis2014, Sanders2017}. Under such a design, studies
typically test whether patterns of rare variation within a gene or
region of interest differ between affected and unaffected subjects using
either burden \citep{Li2008, Madsen2009} or variance-component
\citep{Wu2011} approaches based on an underlying logistic-regression
framework that treats disease status as a simple dichotomous outcome
variable. While such an analysis strategy is commonplace, there may
exist helpful secondary information collected by the study that can
facilitate the creation of a modified outcome variable that is more
refined than the coarse dichotomous outcome typically considered. Use of
this refined outcome variable within the study can reduce heterogeneity
and potentially lead to more powerful analyses.

One valuable source of secondary information often collected in a
case-control study (but rarely utilized) is whether a sample participant
reports a family history of the disease under study. Subjects with a
family history of disease demonstrate different patterns of genetic
variation than their sporadic counterparts. In particular, several
papers have noted that a sample of cases reporting affected relatives
are more enriched for a causal variant than cases without such family
history \citep{TengRisch1999, Zollner2012, Epstein2015} since
\textcolor{blue}{families with multiple affected individuals tend to segregrate more risk variants.}
Likewise, controls with a family history of disease should have elevated
frequency of a causal variant compared to sporadic controls
\citep{Liu2017}. These observations motivate replacement of the standard
dichotomous outcome variable for disease with a more refined variable
that incorporates family-history information into the coding.

In deciding how to refine the variable, we note that we should expect
the frequency of a risk variant to follow a gradient that increases in
frequency from sporadic controls to controls with a family history to
sporadic cases to cases with a family history. One way to exploit this
phenomenon in genetic analysis is to recode the disease variable as a
ordinal cateogorical variable with four possible levels: controls
(\(0\)), controls with a family history (\(1\)), sporadic cases (\(2\)),
and cases with a family history (\(3\)). If family-history information
is unavailable for controls, we instead consider a ordinal categorial
variable with three possible levels: controls (\(0\)), sporadic cases
(\(1\)), and cases with a family history (\(2\)). In either case, this
recoding requires the development of novel methods for rare-variant
analysis that can handle ordinal variables. To fill this gap, we propose
a novel approach that is an extension of our previous GAMuT approach
\citep{Broadaway2016}, which is a nonparametric association test using a
kernel-distance covariance (KDC) framework that can handle
multi-dimensional genotypes and phenotypes. Kernel-based approaches have
found success in rare variant associations due to the natural
incorporation of epistatic effects, and sparsity in the methodology.
Here, we show how GAMuT can model ordinal outcomes in rare-variant
analysis while correcting for confounding covariates such as population
stratification. Furthermore, just like the standard GAMuT, the newly
proposed ordinal GAMuT produces analytical p-values, which facilitates
scaling to genome-wide analyses.

The structure of this paper is as follows: after introducing the ordinal
GAMuT method using the KDC framework
\citep{Gretton2008, Szekely2007, Kosorok2009, Zhang2012, Hua2015}, we
present simulation work to show that leveraging family history
information via ordinal categorical variables can improve power in
rare-variant association tests compared to standard dichotomous modeling
of disease phenotypes that ignore such information, like the burden test
\citep{Li2008, Madsen2009} and Sequence Kernel Association Test (SKAT)
\citep{Wu2011}. Finally, we apply ordinal GAMuT to rare and less-common
variant data from a genome-wide study of craniofacial defects
\citep{Leslie2016, Leslie2016b, Mostowska2018}.

\hypertarget{materials-and-methods}{%
\section{Materials and Methods}\label{materials-and-methods}}

\hypertarget{leveraging-family-information-through-ordinal-phenotype}{%
\subsection{Leveraging Family Information through Ordinal
Phenotype}\label{leveraging-family-information-through-ordinal-phenotype}}

We assume a sample of \(N\) subjects that are genotyped for \(V\) rare
variants in a target gene or region, so that
\(G_j = (G_{j,1}, G_{j,2},\dots, G_{j,V})\) represents the genotypes of
subject \(j\) at \(V\) rare-variant sites in the gene of interest. Note
that \(G_{j,v}\) represents the number of copies of the minor allele
that the subject possesses at the \(v^{th}\) variant. Thus, the matrix
of rare-variant genotypes for the sample is denoted
\(\mathbf{G} \in \mathbb{R}^{N \times V}\).

Let \(\mathbf{Q}\) be an \(N\)-dimensional vector with binary disease
status for \(N\) subjects. That is, \(Q_j=0\) if subject \(j\) is a
control, and \(Q_j=1\) if subject \(j\) is a case. When family history
information is available, we can instead employ a more informative
ordinal phenotype. Assuming family history information is only available
on cases, we can define the ordinal score as \(\tilde{Q}_j\) = \(0\) if
the subject is a control, \(\tilde{Q}_j =1\) if the subject is a case
without family history of the disease, and \(\tilde{Q}_j =2\) if the
subject is a case with family history of the disease. If family history
information is available for controls, we can modify appropriately by
extending the ordinal variable to the case of four categories: controls
without (\(\tilde{Q}_j=0\)) and with family history (\(\tilde{Q}_j=1\)),
and cases without (\(\tilde{Q}_j=2\)) and with family history
(\(\tilde{Q}_j=3\)). The resulting phenotype vector \(\tilde{Q}\) is an
\(N\)-dimensional ordinal vector with disease binary status adjusted for
family history for the \(N\) subjects.

\hypertarget{adjusting-for-covariates}{%
\subsection{Adjusting for Covariates}\label{adjusting-for-covariates}}

After transforming the binary phenotype to ordinal phenotype by
incorporating the family history information, we can account for other
covariates by regressing the phenotypes \(\tilde{Q}_j\) on covariates
\(X_j\) with a cumulative-logit regression model, and use the residuals
in our subsequent rare-variant association test. To illustrate the
cumulative-logit regression model, let \(\tilde{Q}_j\) be an ordinal
response with \(M\) categories, and let \(P(\tilde{Q}_j \leq k)\) be the
cumulative probabilities for \(k=1, \dots, M\). The proportional odds
model \citep{McCullagh1989} is a subclass of cumulative-logit regression
models and it is defined as \[
\logit P(\tilde{Q}_j \leq k | \mathbf{X_j}) = \theta_k - \mathbf{\beta}^T\mathbf{X_j} 
\] for \(k=1,\dots,M-1\). Note that the negative sign is a convention to
guarantee that large values of \(\mathbf{\beta}^T\mathbf{X_j}\) increase
the probability in the larger values of \(k\). In addition, the vector
of intercepts \(\theta = (\theta_1,\dots,\theta_{M-1})\) should satisfy
\(\theta_1 \leq \theta_2 \leq \dots \leq \theta_{M-1}\).

This model is denoted proportional odds because the ratio of the odds of
\(P(\tilde{Q}_j \leq k|\mathbf{X_j})\) and
\(P(\tilde{Q}_{j'} \leq k|\mathbf{X_{j'}})\) do not depend on the
specific category \(k\). That is, \[
\frac{P(\tilde{Q}_j \leq k | \mathbf{X_j})/(1-P(\tilde{Q}_j \leq k | \mathbf{X_j}))}{P(\tilde{Q}_{j'} \leq k | \mathbf{X_{j'}})/(1-P(\tilde{Q}_{j'} \leq k | \mathbf{X_{j'}}))} = \exp(-\mathbf{\beta}^T(\mathbf{X_j}-\mathbf{X_{j'}}))
\] This is also denoted a \textit{parallelism assumption} on
\(\mathbf{\beta}\) \citep{Yee2010}.

Note that for an ordinal response with \(M\) categories, we fit \(M-1\)
logit regression models. Thus, in our particular setting, we have three
categories: controls (\(k=0\)), cases without family history (\(k=1\))
and cases with family history (\(k=2\)), and thus, we will fit 2 models:
\(\logit P(\tilde{Q}_j \leq 0)\) and \(\logit P(\tilde{Q}_j \leq 1)\).
With these models, we estimate the multinomial response probabilities
for each individual. That is, for individual \(j\), we have: \[
\mu_{j,0} = P(\tilde{Q}_j=0),
\mu_{j,1} = P(\tilde{Q}_j=1),
\mu_{j,2} = P(\tilde{Q}_j=2)
\] Thus, the matrix of fitted values (denoted \(\mathbf{M}\)) will be a
\(N \times 3\) matrix where each row sums to 1, and the \(i^{th}\) row
corresponds to the estimated multinomial probabilities for individual
\(i\): \((\hat{\mu}_{j,0},\hat{\mu}_{j,1},\hat{\mu}_{j,2})\). To obtain
the matrix of residuals, we first transform the ordinal response into a
\(N \times 3\) binary matrix (denoted \(\mathbf{I}_{\tilde{Q}}\)) where
the \(i^{th}\) row corresponds to the 3-dimensional vector for
individual \(i\) with three indicator functions, one for each category:
\((I(\tilde{Q}_j=0), I(\tilde{Q}_j=1), I(\tilde{Q}_j=2))\). For example,
if \(\tilde{Q}_j=2\), then the binary vector in the \(j^{th}\) row would
be \((0,0,1)\). As a result, the matrix of residuals \(\mathbf{R}\) will
be the \(N \times 3\) matrix of the difference between the binary matrix
and the matrix of estimated multinomial probabilities:
\(\mathbf{I}_{\tilde{Q}}-\mathbf{M}\). This matrix of residuals will
then be input into the GAMuT framework to enable rare-variant
association testing. The GAMuT framework allows for correlated
phenotypes, and will be described in the following section.
\textcolor{blue}{Given that the GAMuT framework simply takes the residuals of the ordinal model as input, we highlight that this approach does not depend on a specific ordinal model. That is, 
the proportional odds model was chosen for simplicity, and the ordinal GAMuT approach could easily be extended to other ordinal models like continuation ratio with different logit formulations. The only difference would be the estimation of $(\hat{\mu}_{j,0},\hat{\mu}_{j,1},\hat{\mu}_{j,2})$ using a different ordinal model with more general assumptions.}

\hypertarget{gamut-test-of-cross-phenotype-associations}{%
\subsection{GAMuT Test of Cross-Phenotype
Associations}\label{gamut-test-of-cross-phenotype-associations}}

GAMuT tests for independence between the phenotype matrix
\(\mathbf{R}=\mathbf{I}_{\tilde{Q}}-\mathbf{M}\) (the \(N \times 3\)
matrix of phenotype residuals) and \(\mathbf{G}\) (the \(N \times V\)
matrix of multivariate rare-variant genotypes) by constructing an
\(N \times N\) phenotypic-similarity matrix \(\mathbf{Y}\), and an
\(N \times N\) genotypic-similarity matrix \(\mathbf{X}\). These
similarity matrices depend on a user-selected kernel function
\citep{Kwee2008, Schaid2010, Wu2010, Wu2011}. For example, the matrix
\(\mathbf{Y}\) can be modeled with the projection matrix:
\(\mathbf{Y} = \mathbf{R} (\mathbf{R}^T\mathbf{R})^{-1}\mathbf{R}^T\).
Alternatively, if \(\gamma(\mathbf{R}_i,\mathbf{R}_j)\) denotes the
kernel function between subjects \(i\) and \(j\), the linear kernel is
defined as
\(\gamma(\mathbf{R}_i,\mathbf{R}_j)= \sum_{l=1}^L R_{i,l} R_{j,l}\),
which corresponds to the \((i,j)\) entry in \(\mathbf{Y}\): \(Y_{ij}\).
See \citet{Broadaway2016} for more details on other kernel functions to
model pairwise similarity or dissimilarity.

After constructing the similarity matrices \(\mathbf{Y}\) and
\(\mathbf{X}\), we center them as \(\mathbf{Y}_c = \mathbf{HYH}\) and
\(\mathbf{X}_c = \mathbf{HXH}\), where
\(\mathbf{H}=(\mathbf{I}-\mathbf{11}^T/N)\) is a centering matrix
(\(\mathbf{HH} = \mathbf{H}\)),
\(\mathbf{I} \in \mathbb{R}^{N \times N}\) is an identity matrix, and
\(\mathbf{1} \in \mathbb{R}^{N \times 1}\) is a vector of ones. With the
centered similarity matrices (\(\mathbf{Y}_c, \mathbf{X}_c\)), we
construct the GAMuT test statistic as

\[
T_{GAMuT}=\frac{1}{N}\trace(\mathbf{Y}_c\mathbf{X}_c).
\] Under the null hypothesis where the two matrices are independent,
\(T_{GAMuT}\) follows the asymptotic distribution as the weighted sum of
independent and identically distributed \(\chi^2_{(1)}\) variables
\citep{Broadaway2016}. We then use Davies' method \citep{Davies1980} to
analytically calculate the p-value of \(T_{GAMuT}\).

\hypertarget{simulations}{%
\subsection{Simulations}\label{simulations}}

We conducted simulations to show that ordinal GAMuT properly preserves
the type I error and to assess the power of ordinal GAMuT relative to
standard case-control burden \citep{Li2008, Madsen2009} and SKAT
\citep{Wu2011} tests that do not account for family history information.

For the genetic data, we simulated trios (parents and offspring) with
10,000 haplotypes of 10 kb in size using COSI \citep{cosi}, a coalescent
model that accounts for linkage disequilibrium (LD) pattern, local
recombination rate, and population history for individuals of European
descent. We defined rare variants as those with \(MAF \leq 3\%\). For
the power simulations, we assumed the proportion of causal variants to
be \(15\%\), with effect size for each causal variant given by
\(\frac{\log(C)}{4}|\log_{10}(MAF)|\) plus a Normal noise with mean
\(0\) and variance \(0.1\). We varied \(C=4,6\). This setup defines the
effect size of any given causal variant as inversely proportional to its
MAF, which implies that very rare variants will have a larger effect
size.

For the ordinal phenotype, the proband's probability of disease depended
on the sequence data and the disease prevalence, which we varied as
\(0.01\), or \(0.05\), while the family members' probability of disease
depended on the sequence data and the conditional recurrence risk ratio
(\(\lambda=2,4,8\)) \citep{Epstein2015}. If the proband was unaffected,
we defined the person as a control. If the proband was affected and none
of the parents were affected, we defined the person as a case without
family history. Finally, if the proband was affected and at least of the
parents was affected, we defined the person as a case with family
history.

For each simulated dataset, we generated an equal number of controls,
cases without family history, and cases with family history. We varied
this number among \(N=400, 750, 1000, 1500\). For each simulated
dataset, we applied our ordinal GAMuT method that modeled cases with and
without family history separately. We also applied standard burden and
SKAT tests that combined all cases together without regards to family
history information. For each method, we weighted rare variants using
the weighting scheme recommended by \citet{Wu2011};
\(w_v=Beta(MAF_v,1,25)/Beta(0,1,25)\).

We used the R package \texttt{VGAM} and function \texttt{vglm} to fit
the cumulative-logit regression model with proportional-odds assumption
\citep{Yee2010}, and use the resulting residuals to construct the
phenotypic similarity matrix input in the GAMuT package
\citep{Broadaway2016}.

\hypertarget{analysis-of-pittsburgh-orofacial-cleft-multiethnic-gwas}{%
\subsection{Analysis of Pittsburgh Orofacial Cleft Multiethnic
GWAS}\label{analysis-of-pittsburgh-orofacial-cleft-multiethnic-gwas}}

Orofacial clefts (OFCs) such as cleft lip (CL), cleft palate (CP), and
cleft lip with cleft palate (CLP) are among the most common birth
defects in humans with prevalence between 1 in 500 and 1 in 2,500 live
births \citep{Tessier1976, Mossey2009}. Extensive recent studies
identified common nucleotide variants associated with orofacial clefts,
such as 1p22.1, 2p24.2, 3q29, 8q24.21, 10q25.3, 12q12, 16p13.3, 17q22,
17q23, 19q13, and 20q12
\citep{Birnbaum2009, Grant2009, Beaty2010, Mangold2009, Wolf2015, Leslie2016, Leslie2016b, Mostowska2018}.
However, the role of rare genetic variation in OFCs is still underway.

The Pittsburgh Orofacial Cleft Multiethnic GWAS
\citep{Leslie2016, Leslie2016b} seeks to identify genetic variants that
are associated with the risk of OFCs. This dataset includes a
multi-ethnic cohort with 11,727 participants from 13 countries from
North, Central or South America. Asia, Europe and Africa. Most of the
participants were recruited as part of genetic and phenotyping studies
coordinated by the University of Pittsburgh Center for Craniofacial and
Dental Genetics and the University of Iowa. The study cohort includes
OFC-affected probands with their family members, and controls without
history of OFC. Affection status consists of cleft lip (CL) with or
without palate (CL/P).

We performed standard data cleaning and quality control (see
\citet{Leslie2016}). We analyzed only Caucasian participants, and we
kept rare variants with MAF in \((0.001,0.05)\) and genotype call rate
greater than \(95\%\).

The final sample consisted of 1411 individuals, among which there were
835 controls, 309 cases without family history and 267 cases with family
history. We did not include any covariates except for 5 principal
components of ancestry (see \citet{Leslie2016} for the details on the
Principal Components Analysis). We applied ordinal GAMuT using linear
kernel to measure pairwise phenotypic similarity. We also ran SKAT and
burden tests, with the typical weights defined in \citet{Wu2011}. For
GAMuT, we used a weighted linear kernel (with the weighting scheme in
\citet{Wu2011}) to measure pairwise genotypic similarity.

\textbf{Data availability statement } The URLs for software:
\url{https://github.com/crsl4/ordinal-gamut} and
\url{http://www.genetics.emory.edu/labs/epstein/software}. The dataset
title and accession number for dbGaP are ``Center for Craniofacial and
Dental Genetics (CCDG): Genetics of Orofacial Clefts and Related
Phenotypes'', ``dbGaP Study Accession: phs000774.v2.p1''.

\hypertarget{results}{%
\section{Results}\label{results}}

\hypertarget{type-i-error-simulations}{%
\subsection{Type I Error Simulations}\label{type-i-error-simulations}}

Figure \ref{fig:qq} shows the quantile-quantile (QQ) plots of 10,000
null simulations with different subjects per group, target disease
prevalence, and \(\lambda\) values. We show the comparison with ordinal
GAMuT, SKAT and burden test. All methods compared properly control the
type I error.

\begin{figure}

{\centering \includegraphics[width=0.8\linewidth]{figures/scenario1} \includegraphics[width=0.8\linewidth]{figures/scenario3-2} \includegraphics[width=0.8\linewidth]{figures/scenario5} \includegraphics[width=0.8\linewidth]{figures/scenario7} 

}

\caption{Q-Q plots of p-values for gene-based tests of rare variants for three methods: burden test, SKAT, and the ordinal GAMuT here proposed. Simulated datasets (10,000) assumed a 10kb region and rare variants defined as those with MAF <3\%. \textbf{Top:} 750 subjects per group,  disease prevalence of 0.01 and $\lambda=2$. \textbf{Middle Top:} 750 subjects per group, disease prevalence of 0.05 and $\lambda=2$. \textbf{Middle Bottom:} 750 subjects per group, disease prevalence of 0.01 and $\lambda=4$. \textbf{Bottom:} 750 subjects per group, disease prevalence of 0.05 and $\lambda=4$.}\label{fig:qq}
\end{figure}

\hypertarget{power-simulations}{%
\subsection{Power Simulations}\label{power-simulations}}

Now, we compare the power of ordinal GAMuT with SKAT and burden test
(Figure \ref{fig:power-plot}). The power was estimated by computing the
proportion of p-values less than the significance level
(\(\alpha=5 \times 10^{-5}\) for effect sizes of 4 and 6) out of 1000
replicates per scenario and model. For these power simulations, we use
different effect sizes in figures (\(C=4,6\)). Columns refer to the
conditional recurrence risk ratio \(\lambda=2,4,8\), and rows refer to
disease prevalences \(0.01,0.05\) . We compare the empirical power for
sample sizes of \(N=400,750,1000,1500\) subjects per group. First, we
note that we observe an increased number of causal variants in cases
with family history compared to controls (see Supplementary Materials).
Our method (ordinal GAMuT) outperformed the burden test
\citep{Li2008, Madsen2009} and SKAT \citep{Wu2011}, with power
increasing as sample size, recurrence risk, and effect size increased.
Our method is more powerful given that other methods merge two clearly
distinct groups: cases with and without family history, and thus, they
cannot exploit the information present in the enrichment of causal
variants in the cases with family history. Our ordinal approach models
reality better by explicitly separating these two groups that have
distinct genetic characteristics.

\begin{figure}

{\centering \includegraphics{ordinal-gamut_files/figure-latex/power-plot-1} 

}

\caption{Power for gene-based testing comparing three methods: burden test (blue, square), SKAT (green, triangle) and ordinal GAMuT (red, circle). We compared two disease prevalences 0.01, 0.05 (rows), different conditional recurrence risk ratio $\lambda=2,4,8$ (columns).}\label{fig:power-plot}
\end{figure}

\hypertarget{analysis-of-pittsburgh-orofacial-cleft-multiethnic-gwas-1}{%
\subsection{Analysis of Pittsburgh Orofacial Cleft Multiethnic
GWAS}\label{analysis-of-pittsburgh-orofacial-cleft-multiethnic-gwas-1}}

We applied our method to a Pittsburgh Orofacial Cleft (POFC) Multiethnic
GWAS \citep{Leslie2016}, \citep{Leslie2016b} with 1,411 Caucasian
subjects (267 cases with family history of clefting (up to third degree
relatives), 309 cases without family history and 835 controls) and
61,671 variants used for annotation with Bystro \citep{Kotlar2018}. We
filtered rare variants with MAF \([0.001,0.05]\), and filtered genes to
having minimum 4 rare variants, which resulted in 5,137 gene tests. We
tested the association between the 5,137 genes and CL or CL/P status,
adjusting for principal components for population structure. We compared
our results (ordinal GAMuT) with the burden test and SKAT approach.
Neither of the methods show any p-value inflation (Fig.
\ref{fig:cleft-qq}).

None of the methods identified any genes significantly associated with
CL/P. However, ordinal GAMuT identified one gene (GRHL2) on chromosome 8
that passes the suggestive significance threshold (Fig.
\ref{fig:cleft-manhattan}). GRHL2 is in the same gene family as GRHL3,
which is a transcription factor that causes syndromic forms of clefting
and is associated with nonsyndromic clefting in other GWAS
\citep{Leslie2016, Leslie2016b, Carpinelli2017, PeyrardJanvid2014}.

\begin{figure}
\centering
\includegraphics{ordinal-gamut_files/figure-latex/cleft-qq-1.pdf}
\caption{\label{fig:cleft-qq}Q-Q plots of p-values for gene-based tests of
rare variants for three methods: burden test \citep{Li2008, Madsen2009},
SKAT \citep{Wu2011} and the ordinal GAMuT here proposed in the GWAS of
Pittsburgh Orofacial Cleft Multiethnic.}
\end{figure}

\begin{figure}
\centering
\includegraphics{ordinal-gamut_files/figure-latex/cleft-manhattan-1.pdf}
\caption{\label{fig:cleft-manhattan}Gene-based test on Pittsburgh Orofacial
Cleft (POFC) Multiethnic GWAS using burden, SKAT and ordinal GAMuT
approach. Manhattan plots for each of the three tests. Red line:
genome-wide significance level (\(-\log_{10}(0.05/5137) = 5.0117\)).
Blue line: suggestive level (\(-\log_{10}(1 \times 10 ^{-4}) = 4\)).}
\end{figure}

\hypertarget{discussion}{%
\section{Discussion}\label{discussion}}

Standard GWAS methods for case-control studies usually define a disease
outcome as a dichotomous phenotype. This phenotype ignores family
history of disease, even if this information is available in the dataset
at hand. Given that cases with a family history of disease can be
enriched for risk variation relative to sporadic cases and may represent
a source of case heterogeneity, incorporating family history is expected
to increase power to detect genetic variants associated with disease.

We introduce an extension to the GAMuT method \citep{Broadaway2016} to
incorporate family information to enhance case-control association
studies. This approach converts the usual binary phenotype of
case-control status into an ordinal phenotype with three levels: cases
with family history, cases without family history and controls, and it
allows adjustment for covariates. Even though we do not include controls
with family history, this ordinal approach can easily be extended to the
case of four categories: controls with and without family history, and
cases with and without family history by considering an ordinal
phenotype with 4 levels.
\textcolor{blue}{It is common practice to exclude controls with a family history of disease because it is believed that the inclusion of such controls would reduce the power of a standard case-control approach, and also, there are not many software alternatives that can leverage information from such controls. Our ordinal methodology will allow for an easy inclusion of controls with family history, and we showed here that this inclusion increases power.}

Finally, just as the standard GAMuT test, the ordinal GAMuT obtains
analytic p-values from Davies' method \citep{Davies1980} which is
computationally efficient, allowing the analysis of datasets in the
genomic scale.

Simulation studies of rare variant sets showed that our ordinal GAMuT
method is more powerful compared to usual gene-based tests like burden
test \citep{Li2008, Madsen2009} and SKAT \citep{Wu2011}, possibly due to
the fact that subjects with family history are more enriched for rare
causal variants. Applying our method to Pittsburgh Orofacial Cleft
Multiethnic GWAS \citep{Leslie2016, Leslie2016b}, we identified a gene
(GRHL2) (not previously reported) to suggestively associate with cleft
lip and palate phenotypes. GRHL2 is in the same gene family as GRHL3,
which is a transcription factor that causes in syndromic forms of
clefting and was found to be associated with nonsyndromic clefting in
GWAS \citep{Leslie2016, Leslie2016b, Carpinelli2017, PeyrardJanvid2014}.
Burden and SKAT on these same phenotypes (figure
\ref{fig:cleft-manhattan}) failed to identify any significant or
suggestive genes. Among the weaknesses of the proposed method, extra
care should be taken if there is a small cell count of cases with family
history in the dataset, or in highly unbalanced dataset in which one of
the categories is highly dominant in frequency compared to the other
categories.

We envision two main future extensions of ordinal GAMuT: 1) to include
information of more nuanced definitions of family history, and 2) to use
disease liability as continuous phenotype instead of a categorical
phenotype. Regarding disease liability, options for enhanced outcome
variables could involve conditional means from liability-threshold
models which have the potential to increase the power to detect genetic
variants that are associated with disease risk. In fact, the popularity
of proportional odds can be related to its connection to a linear
regression model on a continuous latent response (e.g.~the liability
score). That is, the ordinal variable \(Y\) is obtained from a latent
continuous variable \(Z\) by \(Y=k\) if \(c_{k-1} < Z \leq c_k\). Thus,
current ordinal GAMuT which utilizes proportional odds model has a
natural extension into linear regression of the latent phenotype of
disease liability. In the liability scale, family history can then be
modeled as joint liability scores with a covariance matrix defined by
the heritability of the disease.

\textcolor{blue}{We can note that the proportional odds assumption can be relaxed due to the flexibility of the multiple phenotypes entering GAMuT. That is, the residuals could be obtained from the proportional odds model (as exemplified here) or from the ordinal continuation ratio model with different logit formulations. The ordinal GAMuT methodology does not rely on a specific type of ordinal model.}

Finally, ordinal GAMuT is not restricted to rare genetic variants.
Similar analysis could be performed for gene-based analysis of common
variation.

\textbf{Acknowledgements:} Data for the Orofacial Cleft Multiethnic GWAS
comes from samples provided by Kaare Christensen (University of Southern
Denmark), Frederic W.B. Deleyiannis (University of Colorado School of
Medicine, Denver), Jacqueline T. Hecht (McGovern Medical School and
School of Dentistry UT Health at Houston), George L. Wehby (University
of Iowa), Seth M. Weinberg (University of Pittsburgh), Jeffrey C. Murray
(University of Iowa) and Mary L. Marazita (University of Pittsburgh).
This work was supported by NIH grants GM117946 {[}MP,DG{]}, HG007508
{[}MP{]}, R00-DE025060 {[}EJL{]}, X01-HG007485 {[}MLM{]}, R01-DE016148
{[}MLM, SMW{]}, U01-DE024425 {[}MLM{]}, R37-DE008559 {[}JCM, MLM{]},
R21-DE016930 {[}MLM{]}, R01-DE012472 {[}MLM{]}, R01-DE011931 {[}JTH{]},
R01-DE011948 {[}KC{]}, U01-DD000295 {[}GLW{]}; NIH contract to the Johns
Hopkins Center for Inherited Disease Research: HHSN268201200008I.

\newpage

\hypertarget{supplementary-material}{%
\section*{Supplementary Material}\label{supplementary-material}}
\addcontentsline{toc}{section}{Supplementary Material}

\hypertarget{enrichment-of-causal-variants}{%
\subsection*{Enrichment of Causal
Variants}\label{enrichment-of-causal-variants}}
\addcontentsline{toc}{subsection}{Enrichment of Causal Variants}

In Figure \ref{fig:causalvar}, we show that, as expected, the average
number of causal rare variants is greater for the cases with family
history, followed by cases without family history, and lastly for
controls. This simulated dataset comprises of 1000 controls, 1000 cases
without family history, and 1000 cases with family history for three
levels of conditional recurrence risk ratios (columns:
\(\lambda=2,4,8\)) and 2 siblings as family history. The effect size was
set as \(C=2\).

\begin{figure}[H]

{\centering \includegraphics{ordinal-gamut_files/figure-latex/causalvar-1} 

}

\caption{Average of 1000 simulations of number of causal rare variants (left) and probability of disease (right) in proband for three groups: controls, cases without family history, and cases with family history under two disease prevalences (red=0.01, blue=0.05), with one (top) or two (bottom) siblings, and three conditional recurrence risk ratios as columns.}\label{fig:causalvar}
\end{figure}

\bibliography{ordinal-gamut.bib}


\end{document}
